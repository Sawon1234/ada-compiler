\documentclass[12pt]{article}%
\usepackage{amsfonts}
\usepackage{fancyhdr}
\usepackage{comment}
\usepackage[a4paper, top=2.5cm, bottom=2.5cm, left=2.2cm, right=2.2cm]%
{geometry}
\usepackage{times}
\usepackage{graphicx}

\usepackage{hyperref}

\usepackage{amssymb}
\usepackage{graphicx}%

\begin{document}

\title{Compilers - Assignment 2 - Parser}
\author{Ayush Sekhari (12185) \\ Vasu Sharma (12785) \\ Margaux Dorido}
\date{\today}
\maketitle
\section{Design Specifications}
We have the following language specifications for our Lexer: \\
Source Language: ADA \\
Target Language: MIPS \\
Implementation Language: Python \\ 
Pythons PLY library was used to build the Lexer and Parser. 

\section{Building the Lexer}
The Lexer has been implemented in python and can be executed as:
\begin{enumerate}


\item  Change directory to the directory containing lexer.py
\item  python ./lexer.py ada\_program\_name.adb
example: python ./lexer.py test1.adb

\end{enumerate}

\subsection{Lexer Constraints}
Our Lexer doesn’t handle the following cases:

\begin{enumerate}
\item Operator overloading is disallowed by our Lexer due to ambiguity in classi-
fying the redefined operator as string or identifier as operator overloading
requires the operator to be put in double quotes which makes it impossible
to distinguish it from other strings.

\item Our Lexer doesn’t allow user to use reserved types as identifier names for
example I can’t have an identifier named String or Integer etc. This has
been done as it creates the ambiguity as whether to identify as a data type
or an identifier.

\item Ada has some non-standard data types too which are defined within certain libraries. We are not creating separate token names for these non-
standard data types. We have however handled standard data types like
Integer, String etc.

\item In the representation of numbers in the exponential form, our Lexer returns
the value of a number in exponential form in the normal numeric decimal
representation. This leads to overflow errors in case the exponential number is too large. example 19E200 is out of bounds of any data type in Ada
and hence causes an overflow causing a garbage value to be returned. We
haven’t handled such issues.
\end{enumerate}

\section{Building the Parser}
They have been commented out in the file {\tt reserved\_tokens.py} \\

The Parser has been implemented in python and can be executed as:
\begin{enumerate}
\item  Change directory to the directory containing parser.py
\item   ./parser.py ada\_program\_name.adb
example: ./parser.py test1.adb
\item A few test files have been listed in the test folder
\item The parse tree can be generated using the following command \\
{\tt dot -Tpng file\_name.dot -o Image\_name.png}

\end{enumerate}
\subsection{Parser Contraints}
\begin{enumerate}
\item We have removed the token Ada\_Keyword from the lexer, We would ensure that this token newly identified as an IDENTIFIER has the value "ada" in the semantic analysis stage. 
\item We needed to remove a few token from our lexer definitions which were difficult to handle as we were not certain of their applications and meanings while building the parser and writing the language grammar. Most of them were attributes. 
\item We have removed tokens for the INTEGER\_TYPE and FLOAT\_TYPE. They are identified as IDENTIFIERS for now.

\end{enumerate}

\end{document}